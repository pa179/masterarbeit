% Meyer: Zusammenfassung = Fazit, keine Wiederholung der bisherigen Diskussion; Bewertung

Nachdem bis zu diesem Punkt nun alle wichtigen und geplanten Schritte des Entwicklungsprozesses abgeschlossen wurden, können die Ergebnisse nun noch einmal in knapper und prägnanter Form zusammengetragen werden. 


\subsection{Arbeitsergebnis}

Aufbauend auf der zu Beginn beschriebenen Problemstellung konnte in dieser Arbeit der gesamte Entwicklungsprozess von der Analyse und Planung einer passenden Lösung bis hin zur prototypischen Umsetzung und ausführlichen Analyse und Bewertung der Ergebnisse durchgeführt werden. Dabei konnten die wichtigen Facetten und Einschränkungen des Problemfelds herausgearbeitet und analysiert werden, sodass darauf aufbauend eine Recherche und Erarbeitung von möglichen Lösungsansätzen durchgeführt werden konnte. Mögliche Optimierungspotenziale konnten so herausgearbeitet werden und verschiedene passende und gängige Verfahren bzw. Ansätze zur Lösung ermittelt und diskutiert. Aufgrund der Komplexität des Anforderungen hat sich dabei allerdings herausgestellt, dass eine einzige, umfassende Lösung nicht einfach zu finden ist. Daher wurde die Entscheidung getroffen, zwei wesentliche und erfolgversprechende Optimierungspotenziale zu nehmen und parallel entsprechende Algorithmen zur Optimierung zu entwickeln. Dafür wurden jeweils konkrete alternative Lösungsmethoden erarbeitet und auch implementiert. Im Anschluss daran konnte ein ausführlicher und quantitativer Vergleich der Lösungen anhand verschiedener als sinnvoll erachteter Metriken durchgeführt werden. Anhand dessen konnten zum einen die Verfahren selbst auf ihre Funktionsfähigkeit getestet werden, zum anderen war es so möglich nachzuweisen, dass eine Verbesserung der Ausgangssituation möglich war und auch wie hoch diese ausfällt. Insgesamt konnte so gezeigt werden, dass eine Optimierung der Ausgangssituation prinzipiell möglich ist. Die prototypisch entwickelten Algorithmen wurden auf ihre Praxistauglichkeit untersucht und die Tests konnten auch darlegen, unter welchen Umständen sie besonders gut funktionieren. Es konnte außerdem anhand praktischer Beispiele demonstriert werden, welche weiteren Aspekte berücksichtigt werden sollten und auch welche Einschränkungen es gibt, die kaum lösbar sind oder wo es Verbesserungspotenziale für eine mögliche Weiterentwicklung gibt.

\subsection{Fazit}
% Bezug auf Einleitung; Was wurde wie gut umgesetzt; ...

Im Verlauf dieser Arbeit konnte eine ganze Reihe wichtiger Ergebnisse erzielt werden. Zunächst einmal wäre da die ausführliche Analyse des Problemfelds zu nennen. Sie bietet nicht nur eine gute Grundlage für die anschließend durchgeführte Prototypenentwicklung, sondern fasst Problematiken und Herausforderungen übersichtlich zusammen, sodass sie auch die Basis für eine möglicherweise folgende Weiterentwicklung und Implementierung im realen Umfeld sein kann. In der anschließenden Konzeption wurden einige bedeutende Möglichkeiten zur technischen Umsetzung recherchiert, auf ihre Anwendbarkeit hin untersucht bzw. diskutiert und bei erfolgversprechendem Ergebnis wurden einige Varianten mit Blick auf die Anforderungen konzipiert. Die Implementierung wurde nach der ausführlicheren und spezifischen Diskussion rudimentär und prototypisch gehalten. Es ging hier darum, zu zeigen dass die Algorithmen funktionieren und an ihnen quantitative Tests durchführen zu können. Zum realen Einsatz ist hier sicherlich noch sehr viel Arbeit nötig, wenn nicht sogar eine komplette Neuimplementation. Ein Prototyp war hier aber auch nur das gesteckte Ziel, welches erreicht werden sollte und wurde. Dementsprechend ist die Implementierung auch nur bedingt auf eine gute Weiternutzung und mehr auf eine sinnvolle Aufteilung für Versuche und die Evaluation ausgelegt worden. Zum Abschluss konnten in der Evaluation noch eine ganze Menge gute Erkenntnisse gewonnen werden. Es konnte nachgewiesen werden, dass die Konzepte und Umsetzungen grundsätzlich funktionieren. Es gab kein spezifisches Optimierungsziel, in den angestellten Versuchen konnte aber eine ungefähre Höhe der Verbesserung gezeigt werden. Außerdem konnten die optimalen Arbeitsbereiche und Grenzen der Implementierungen deutlich gemacht werden. So konnte gezeigt werden, was bereits gut funktioniert und an welchen Stellen noch Nachbesserungsbedarf besteht.

Abschließend lässt sich sagen, dass die gesteckten Ziele erreicht werden konnten. Wie für eine solche Entwicklungsarbeit in einem noch nicht weiter erforschten Problemfeld zu erwarten war, haben sich dabei auch Schwierigkeiten und Herausforderungen gezeigt, sodass natürlich keine komplett geradlinige Entwicklung mit perfektem, vorzeigtbarem Ergebnis entstanden ist. Alle Teilergebnisse haben gezeigt, dass eine genaue Umsetzung der Anforderungen kaum möglich ist. Es wird immer noch abzuwägen sein, in wie weit die bestehenden Prozesse angepasst werden können und dürfen, sodass sie sich sinnvoll optimieren lassen und dass auch eine softwarebasierte Lösung anwendbar ist. Sicher ist, dass sich eine Optimierung nur vornehmen lässt, wenn es grundlegende Anpassungen gibt. Eine perfekte und maßgeschneiderte Lösung, welche einfach in den bestehenden Ablauf integriert werden kann ist einfach nicht umsetzbar. Eine abschließende Einschätzung ist in diesem Rahmen leider nicht möglich, da hier eine sehr große Verflechtung mit anderen Bereichen besteht. Die Verfahren seitens der Speditionen spielen hier eine Rolle, aber auch die Schnittstelle von den angelieferten Waren zu den Verladeeinheiten auf die Schiffe. Hier ist es sehr schwer, anhand der hier erstellten Konzepte eine gute Aussage zur Machbarkeit von den beteiligten Parteien zu erhalten.



\subsection{Ausblick}
% Verbesserungsmöglichkeiten; Was könnte noch nach dem Prototypen weiterentwickelt werden?

Um das hier betrachtete Problem auch in der Realität anzugehen und möglicherweise sogar die hier entwickelten Konzepte und Prototypen einsetzen zu können, ist sicherlich noch eine ganze Menge Arbeit nötig. Zunächst einmal sollten die hier gewonnenen Erkenntnisse einen guten Anhaltspunkt dafür geben, was möglich ist und welche Punkte besonderer Aufmerksamkeit bedürfen. Deshalb müsste zunächst einmal grundlegend von der Prozessseite her festgelegt werden, welche Anpassungen im gesamten Ablauf am Terminal gemacht werden können und dürfen. Anhand dessen wird dann auch zu entscheiden sein, welches der beiden Optimierungskonzepte real überhaupt einsetzbar ist oder ob basierend auf den hier gezeigten Ideen sogar eine ganz andere Variante gewählt werden muss. Im Anschluss daran muss dann natürlich die komplette Einbettung der Algorithmen in das bestehende System vorgenommen werden. Je nachdem, ob dies in dem hier zu Beginn analysierten System erfolgen soll oder ob vielleicht eine gnaz andere Software verwendet wird, entscheidet sich ob Code wiederverwendet werden kann oder ob eine komplette Neuimplemnentierung sinnvoller ist. Dies wäre dann auch vom jeweiligen System und der dort genutzten Programmiersprache abhängig. Dieses System wäre dann auch grundlegend anzupassen. Die Erfassung der hier als unbedingt notwendig erachteten Daten, aber auch die Art und Weise, wie Slots ausgewählt werden und ein Weg, wie später festgelegte Zeitpunkte kommuniziert werden müssen bedacht werden.

Auch bezüglich der konkreten Implementierungen gibt es sicherlich noch einiges Verbesserungspotenzial. So sind die verwendeten Zeitangaben eher grobe Schätzungen und Werte zur Demonstration als wirklich aus der Praxis nachweislich gute Werte, da hier einfach keine guten und verlässlichen Quellen gefunden wurden. Hier wäre es definitiv sinnvoll, eine Überprüfung bzw. Verbesserung der Schätzung durchzuführen und möglicherweise sogar verbesserte Methoden und Wege zur Berechnung anzuwenden. Die einfache Annahme von benötigten Zeiten pro Ladegut bzw. feste Auf- und Abbauzeiten bilden die Realität möglicherweise nicht ganz zuverlässig ab. Sollte einer der beiden entwickelten Algorithmen weiter genutzt werden so gibt es auch dort jeweils noch einige konkrete Ideen zur Weiterentwicklung bzw. Verbesserung. 
Der Algorithmus 1 hat einen sehr vereinfachten Satz von Anforderungen genutzt, um Aussichten auf eine erfolgreiche Umsetzung zu behalten. Nachdem dies nun nachgewiesen wurde, wäre es denkbar hier noch Arbeit in die Erweiterung zu stecken. Vielleicht werden nicht immer alle Ressourcen für die ganze Abfertigungszeit benötigt. Außerdem wurden allgemein oder längerfristig benötigte Resourcen, wie die Abstellfläche im Hafen noch gar nicht berücksichtigt, auch dies ließe sich sicherlich noch integrieren. Wenn sich noch sehr gute Ideen bezüglich anderer heuristischer Methoden zur Verteilung ergeben, könnten auch diese noch umgesetzt werden, die bisher Ausprobierten erscheinen aber bezüglich der erreichten Auslastung als gar nicht so schlecht, sodass diese schon sehr gut funktionieren. 
Der Algorithmus 2 mit den Traveling Salesman Problem ist natürlich ganz klar von dem genutzten Lösungsverfahren abhängig. Hier hatte sich gezeigt, dass das perfekte Verfahren noch nicht implementiert wurde. Auch wenn das Simualted Annealing Verfahren unter den hier genutzten Bedingungen sicherlich sehr gute Ergebnisse liefert, könnte man hier noch Zeit in die effizientere Implementierung eines exakten Algorithmus stecken, um so in annehmbarer Zeit gut Ergebnisse auch für größere Graphen erzielen zu können. Auch in diesem Algorithmus wurden viele Anforderungen zunächst nicht berücksichtigt, um die grundsätzlichen Ideen auf ihre Machbarkeit zu prüfen. Eine Erweiterung des Konzepts könnte hier durchaus umsetzbar sein. Beispielsweise die Implementierung verschiedener möglicher Ladehilfsmittel pro Kategorie, sodass auch hier noch einmal eine Wahl und Optimierung bezüglich des am besten geeigneten Fahrzeugs stattfinden kann. 
