\todo[inline]{TODOs: -Prüfen, ob alle Figures auf im Text referenziert und besprochen werden; - Stapler/Gabelstapler einheitlich; - Organisation bei der Arbeit Kanban o.ä. beschreiben?, Allgemein Methodik u.a. Interviews geführt}

%Das folgende Kapitel soll zunächst eine Einführung in die in dieser Arbeit betrachtete Problemstellung, sowie in Ziele und Ansätze zur Umsetzung geben.

In Zeiten von globaler Wirtschaft mit weltweitem Handel ist die Schifffahrt ein zentraler Pfeiler und wichtiger Bestandteil der Lieferketten zwischen entfernten Ländern und Kontinenten. Mit stetig steigender Menge von ausgetauschten Gütern und gleichzeitig wachsendem Preisdruck wird es immer wichtiger, die gesamte Prozesskette so schnell und günstig wie möglich zu gestalten. Dies gilt insbesondere auch für die Infrastruktur in den Häfen. 

\subsection{Problemstellung}

Die vorliegende Arbeit beschäftigt sich mit einem ganz konkreten und aus der Praxis stammenden Thema im dem Bereich der Hafenlogistik. Zum An- und Abtransport von Waren, die noch verschifft werden sollen oder bereits auf diesem Wege geliefert wurden, wird neben dem Güterverkehr auf der Schiene vor allem der LKW als bedeutendes Transportmittel genutzt \todo{Quellen? Evtl. Graphen zur Verteilung bei Analyse des Problemfelds, um die Bedeutung des LKW deutlich zu machen}. Im Gegensatz zur Containerverladung, welche aufgrund der einheitlichen Größen schon sehr standardisiert und optimiert ist, soll es in dieser Arbeit um die Verladung von Stückgütern von und auf LKW gehen. Diese Arbeit ist in Zusammenarbeit mit der HEC GmbH entstanden und beschäftigt sich deshalb auch mit einem Problem, welches in aktuellen Projekten für die Häfen in Hamburg und Bremerhaven\todo{Häfen benennen} aufgetreten ist. Momentan gibt es dort eher einfach gehaltene Softwaresysteme, über die LKW Fahrer ihre geplante Lieferung bzw. Abholung in festgelegten Zeitslots avisieren können. In jeden Zeitrahmen können allerdings mehrere LKW gebucht werden. Die Verteilung und Reihenfolge der Abfertigung läuft dabei nach keinem organisierten System, sondern eher zufällig bzw. nach Reihenfolge der Buchung. Dies führt allerdings dazu, dass die Abfertigung von Zeit und Aufwand nicht besonders optimal abläuft. Eine Berücksichtigung bspw. von benötigter Lademaschine, Verfügbarkeit von Personal oder Lagerplatz könnte hier zu Zeit- und somit Geldersparnissen bei Terminal und LKW Fahrer führen. 

\subsection{Ziele der Arbeit}

Ziel dieser Arbeit soll es sein, eine softwarebasierte Lösung bzw. Optimierung für das zuvor beschriebene und relativ spezifische Problem zu finden. Ganz konkret heißt dies, einen Algorithmus zu entwickeln, welcher die benötigte Abfertigungszeit pro LKW durch geschickte Planung reduzieren kann bzw. so dafür sorgt, dass mehr LKW in gleicher Zeit bearbeitet werden können. Da es sich um einen durchaus komplexen Kontext handelt, der auch in vielen Details nicht direkt mit anderen Terminals zu vergleichen ist, wird ein erster wichtiger Teil dieser Arbeit die Analyse der Dimensionen des Problems mit anschließender Anforderungsanalyse sein. Dabei geht es darum, alle Regeln, Abhängigkeiten und Parameter zu verstehen, die in einer Optimierung bedacht werden sollten. Dies wird allerdings sehr komplex sein und auch einige schlecht oder gar nicht planbare Unsicherheiten beinhalten. Ziel dieser Arbeit wird deshalb keine hoch genaue und technische Optimierung des Problems sein. Viel mehr soll hier der Gesamtkontext bearbeitet werden, verschiedene technische Möglichkeiten zur Umsetzung abgewogen werden und anschließend anhand einer prototypischen Planung und Umsetzung die Machbarkeit der theoretischen Betrachtung gezeigt werden. Auch wenn nicht alle Aspekte praktisch berücksichtigt werden können, wird der gesamten Entwicklungsprozess viele Erkenntnisse zu Problemen, Herausforderungen und Möglichkeiten liefern. Auch dies ist ein wichtiges Ergebnis für eine mögliche anschließende reale Umsetzung.

\subsection{Lösungsansatz}

Momentan wird hier in der Regel ein sehr statisches System zur Buchung von Zeitslots durch die LKW Fahrer bzw. deren Speditionen verwendet. Unter Angabe einiger Details zur gelieferten, bzw. zu holenden Ware, können feste Zeiträume gebucht werden, in denen die LKW in den Hafen einfahren und ihre Lieferung durchführen dürfen. Die Idee ist vor allem, eine Optimierung der Abfertigung innerhalb eines Zeitslots zu finden. Die Vergabe der Slots selbst ist dabei wenig variabel. Im Gegensatz zu vielen anderen Anwendungsbereichen steht hier hinter den gelieferten Waren kein sich wiederholender und somit gut planbarer Produktionsprozess und auch die Art der Waren variiert stark. Der Ladeprozess selbst hängt von sehr vielen Faktoren ab, z.B. Verfügbarkeit von Ladehilfsmitteln, Personal oder Stellplätzen. 

Nach einer ausführlichen Erfassung der Rahmenbedingungen, wird es darum gehen, konkrete Anforderungen an eine Optimierung zu formulieren. Ein festgelegter Satz von wirklich wichtigen Eingabedaten ist der erste Schritt für eine sinnvolle Optimierung. Das Hauptziel dieser Arbeit ist es dann, technische Möglichkeiten zur Lösung zu planen und umzusetzen. Die Herausforderung ist es sein, alle Anforderungen zu berücksichtigen und gleichzeitig den Zeitbedarf pro abgefertigtem LKW möglichst weit zu verringern bzw. möglichst viele zusätzliche LKW bearbeiten zu können. Dies ist kein geradliniger Prozess mit einer einzigen, korrekten Lösung, sodass diese Arbeit auch dazu dienen soll, verschiedene Ansätze praktisch auszuprobieren und zu vergleichen. Neben der Recherche und Ausarbeitung von vielversprechenden Methoden, wird auch die Evaluation und Bewertung dieser Ansätze ein wesentlicher Aspekt dieser Arbeit sein. Ein quantitativer Vergleich der Ergebnisse zueinander und auch zur Ausgangssituation kann belegen, ob die entwickelten Algorithmen grundsätzlich funktionieren. Sie erlauben aber auch eine Einschätzung zu möglichen Potenzialen und Höhen der Verbesserung.